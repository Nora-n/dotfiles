%%                    Package `pst-dosage.tex'                                 %%
%%                           for                                               %%
%%                   Acid-base titration curves                                %%
%% This program can be redistributed and/or modified under                     %%
%% the terms of the LaTeX Project Public License Distributed                   %%
%% from CTAN archives in directory macros/latex/base/lppl.txt.                 %%
%%                                                                             %%
%%              Manuel LUQUE <manuel.luque27@gmail.com>                        %%
%% calculs d'apr�s l'article de Marc Chapelet :dans le B.U.P. n.668            %%
%% https://bupdoc.udppc.asso.fr/consultation/article-bup.php?ID_fiche=8847     %%
%%%%%%%%%%%%%%%%%%%%%%%%%%%%%%%%%%%%%%%%%%%%%%%%%%%%%%%%%%%%%%%%%%%%%%%%%%%%%%%%%
\def\fileversion{0.99}
\def\filedate{2023/11/21}

\message{`PST-dosage' v\fileversion, \filedate\space (Manuel Luque)}
\csname PSTDosageLoaded\endcsname
\let\PSTDosageLoaded\endinput

% Require PSTricks, pstplot, pst-xkey and multido packages
\ifx\PSTricksLoaded\endinput\else\input pstricks.tex\fi
\ifx\PSTplotLoaded\endinput\else\input pst-plot.tex\fi
\ifx\MultidoLoaded\endinput\else\input multido.tex\fi
\ifx\PSTXKeyLoaded\endinput\else\input pst-xkey.tex\fi
\ifx\PSTnodesLoaded\endinput\else\input pst-node.tex\fi
\edef\PstAtCode{\the\catcode`\@} \catcode`\@=11\relax
\pst@addfams{pst-dosage}
\newcommand\Cadre[1]{\psframebox[fillstyle=solid,fillcolor=gray,linestyle=none,framesep=0]{#1}}
\define@key[psset]{pst-dosage}{CA}{\edef\psk@Dosage@ConcentrationAcide{#1}}
\define@key[psset]{pst-dosage}{CB}{\edef\psk@Dosage@ConcentrationBase{#1}}
\define@key[psset]{pst-dosage}{VA}{\edef\psk@Dosage@VolumeAcide{#1}}
\define@key[psset]{pst-dosage}{VB}{\edef\psk@Dosage@VolumeBase{#1}}
\define@key[psset]{pst-dosage}{pKA}{\edef\psk@Dosage@pKA{#1}}
\define@key[psset]{pst-dosage}{pKB}{\edef\psk@Dosage@pKB{#1}}
\define@key[psset]{pst-dosage}{pKA1}{\@namedef{psk@Dosage@pKA1}{#1}}
\define@key[psset]{pst-dosage}{pKA2}{\@namedef{psk@Dosage@pKA2}{#1}}
\define@key[psset]{pst-dosage}{pKA3}{\@namedef{psk@Dosage@pKA3}{#1}}
\define@key[psset]{pst-dosage}{dpHunit}{\@namedef{psk@Dosage@dpHunit}{#1}}
\define@key[psset]{pst-dosage}{pH1}{\@namedef{psk@Dosage@pH1}{#1}}
\define@key[psset]{pst-dosage}{pHstyle}{%
	\@namedef{psk@Dosage@pHstyle}{#1}}
\define@key[psset]{pst-dosage}{dpHstyle}{%
	\@namedef{psk@Dosage@dpHstyle}{#1}}
\define@key[psset]{pst-dosage}{tangentesstyle}{%
	\@namedef{psk@Dosage@tangentesstyle}{#1}}
\newif\ifPst@dpH
\define@key[psset]{pst-dosage}{dpH}[true]{\@nameuse{Pst@dpH#1}}%
\newif\ifPst@Equivalence
\define@key[psset]{pst-dosage}{Equivalence}[true]{\@nameuse{Pst@Equivalence#1}}%
\newif\ifPst@values
\define@key[psset]{pst-dosage}{valeurs}[true]{\@nameuse{Pst@values#1}}%
\newif\ifPst@tangentes
\define@key[psset]{pst-dosage}{tangentes}[true]{\@nameuse{Pst@tangentes#1}}%
\newpsstyle{redbold}{linecolor=red,linewidth=2\pslinewidth}
\newpsstyle{bluenormal}{linecolor=blue}
\newpsstyle{DarkGray}{linecolor=darkgray}
\psset{CA=0.1,CB=0.1,VB=10,pKA=4.75,VA=10,%
	pKB=4.75,pKA1=2.1,pKA2=7.2,pKA3=12,%
	pHstyle=redbold,dpHstyle=bluenormal,tangentesstyle=DarkGray,%
	dpH=true,dpHunit=1,pH1=5,Equivalence=true,tangentes=false,valeurs=false}
\newcommand\graphpaper{%
	\psset{gridwidth=1\pslinewidth}
	\psgrid[gridlabels=0,subgriddiv=10,subgridwidth=0.1\pslinewidth,subgridcolor=gray,gridcolor=orange](16,14)%
	\psgrid[gridlabels=0,subgriddiv=2,subgridwidth=0.4\pslinewidth,subgridcolor=orange,gridcolor=orange](16,14)%
	\psset{arrowscale=1.5,arrowinset=0.2}%
	\uput[l](0,14){\cadregris{\textsf{pH}}}%
	\psaxes{->}(16,14)}
\newcommand\cadregris[1]{%
	\psframebox[fillstyle=solid,fillcolor=gray,framesep=0,linestyle=none]{\textcolor{white}{#1}}}
\def\Vbmax{25}
\def\myphmax{12.5}
\def\Valeurs{%
/Helvetica findfont
8 scalefont
setfont
/chaine 10 string def
/AfficheValeurs
{ /Tableau exch def
0 1 Tableau length 1 sub
	{ Tableau exch get
		chaine cvs
		show } for
} def}%
\def\Tangentes{%
	% trac� des tangentes
	pH1 calc
	/V1 V def
	/pH2 2 pHE mul pH1 sub def
	pH2 calc
	/V2 V def
	pH1 0.01 add calc
	/V11 V def
	/M1 0.01 V11 V1 sub div def % pente en 1
	/B1 pH1 M1 V1 mul sub def
	/v11 V1 2 sub def
	/ph11 M1 v11 mul B1 add def
	/v12 V1 2 add def
	/ph12 M1 v12 mul B1 add def
	/B2 pH2 M1 V2 mul sub def
	/v21 V2 2 sub def
	/ph21 M1 v21 mul B2 add def
	/v22 V2 2 add def
	/ph22 M1 v22 mul B2 add def
	/BE pHE M1 VE mul sub def
	/vE1 VE 2 sub def
	/phE1 M1 vE1 mul BE add def
	/vE2 VE 2 add def
	/phE2 M1 vE2 mul BE add def
}%
%%%%%%%%% dosage d'un acide fort par une base forte%%%%%%%%%%%
\def\dosageAB{\@ifnextchar[{\pst@dosageAB}{\pst@dosageAB[]}}%
\def\pst@dosageAB[#1]{{%%
			\psset{#1}%
			%\pspicture(16,15)%
			%\psgrid[subgridwidth=0.2\pslinewidth,gridlabels=0pt]%
			\graphpaper
			\uput[-90](\Vbmax,0){$\SI[parse-numbers=false]{V_B}{mL}$}
			\ifPst@dpH
				\psline[linecolor=black]{->}(\Vbmax,0)(\Vbmax,14)
				\uput[0](\Vbmax,14){$\displaystyle\dv{\pH}{V_B}$}
			\fi
			\pnode(! %
			/pHmax
			/CA \psk@Dosage@ConcentrationAcide\space def
			/CB \psk@Dosage@ConcentrationBase\space def
			/VB \psk@Dosage@VolumeBase\space def
			/VA \psk@Dosage@VolumeAcide\space def
			/pH1 \@nameuse{psk@Dosage@pH1}\space def
			/KE 1e-14 def
			KE VA 16 add mul
			CB 16 mul CA VA mul sub
			div
			log neg def
			/pHmin
			CA
			log neg def
			/pHE 7 def
			/VE CA VA mul CB div def
			/calc{/pH ED
				/H 10 pH neg exp def
				/V CA KE H div add H sub
				H CB add KE H div sub
				div VA mul def} def
			\Tangentes
			VE pHE){E}%
			\psclip{\psframe[linestyle=none](\Vbmax,14)}%
			\parametricplot[style=\psk@Dosage@pHstyle]{pHmin}{\myphmax}{%
				t calc
				V  t}%
			\ifPst@dpH
				\parametricplot[style=\psk@Dosage@dpHstyle,plotpoints=400,yunit=\psk@Dosage@dpHunit]{pHmin}{\myphmax}{%
					t calc
					/V1 V def
					%
					t dt add calc
					/V2 V def
					V1 dt V2 V1 sub div
				}%
			\fi
			\ifPst@Equivalence
				\ifPst@values
					\uput[-45](E){\pnode(! \Valeurs
						[/VE= VE /mL /, /pHE= pHE]
						AfficheValeurs 0 0){AA}}
				\fi
				\psdot(E)
				\uput[180](E){E}
			\fi
			\ifPst@tangentes
				{\psset{style=\psk@Dosage@tangentesstyle}
				\psline(!  v11 ph11)(! v12 ph12)
				\psline(!  v21 ph21)(! v22 ph22)
				\psline(!  vE1 phE1)(! vE2 phE2)
				\psdot(!V1 pH1)
				\psdot(!V2 pH2)}%
			\fi
			\endpsclip
			%\endpspicture
		}}%
%%%%%%%% dosage d'un base forte par un acide fort%%%%%%%%%%%%%%%%%%%
\def\dosageBA{\@ifnextchar[{\pst@dosageBA}{\pst@dosageBA[]}}
\def\pst@dosageBA[#1]{{%
			\psset{#1}%
			%\pspicture(16,15)
			\graphpaper
			\uput[-90](\Vbmax,0){$\SI[parse-numbers=false]{V_A}{mL}$}
			\ifPst@dpH
				\psline[linecolor=black]{->}(\Vbmax,0)(\Vbmax,14)
				\uput[0](\Vbmax,14){$\displaystyle\dv{\pH}{V_B}$}
			\fi
			\pnode( ! %
			/pHmin
			/CA \psk@Dosage@ConcentrationAcide\space def
			/CB \psk@Dosage@ConcentrationBase\space def
			/VB \psk@Dosage@VolumeBase\space def
			/VA \psk@Dosage@VolumeAcide\space def
			/pH1 \@nameuse{psk@Dosage@pH1}\space def
			/KE 1e-14 def
			CA 16 mul
			CB VB mul
			sub
			16 VB add
			div
			log neg def
			/pHmax
			14 CB
			log add def
			/pHE 7 def
			/VE CB VB mul CA div def
			/VE VE 100 mul round 100 div def
			/calc{/pH ED
				/H 10 pH neg exp def
				/V KE H div CB sub H sub
				H CA sub KE H div sub
				div VB mul def} def
			\Tangentes
			VE pHE){E}%
			\psclip{\psframe[linestyle=none](\Vbmax,14)}%
			\parametricplot[style=\psk@Dosage@pHstyle]{pHmin}{\myphmax}{%
				t calc
				V t
			}%
			\ifPst@dpH
				\parametricplot[style=\psk@Dosage@dpHstyle,plotpoints=400,yunit=\psk@Dosage@dpHunit]{pHmin}{\myphmax}{%
					t calc
					/V1 V def
					%
					t dt add calc
					/V2 V def
					V1 dt V2 V1 sub div 14 \psk@Dosage@dpHunit\space div add
				}%
			\fi
			\ifPst@Equivalence
				\ifPst@values
					\uput[-45](E){\pnode(! \Valeurs
						[/VE= VE /mL /, /pHE= pHE]
						AfficheValeurs 0 0){AA}}
				\fi
				\psdot(E)
				\uput[180](E){E}
			\fi
			\ifPst@tangentes
				{\psset{style=\psk@Dosage@tangentesstyle}
				\psline(!  v11 ph11)(! v12 ph12)
				\psline(!  v21 ph21)(! v22 ph22)
				\psline(!  vE1 phE1)(! vE2 phE2)
				\psdot(!V1 pH1)
				\psdot(!V2 pH2)}%
			\fi
			\endpsclip
			%\endpspicture%
		}}
%%%%%%%%% dosage d'un acide faible par une base forte%%%%%%%%%%%%%%
\def\dosageAfBF{\@ifnextchar[{\pst@dosageAfBF}{\pst@dosageAfBF[]}}
\def\pst@dosageAfBF[#1]{{%
			\psset{#1}%
			%\pspicture(16,15)
			\graphpaper
			\uput[-90](\Vbmax,0){$\SI[parse-numbers=false]{V_B}{mL}$}
			\ifPst@dpH
				\psline[linecolor=black]{->}(\Vbmax,0)(\Vbmax,14)
				\uput[0](\Vbmax,14){$\displaystyle\dv{\pH}{V_B}$}
			\fi
			%\pstVerb{
			\pnode(!
			/pHmax
			/CA \psk@Dosage@ConcentrationAcide\space def
			/CB \psk@Dosage@ConcentrationBase\space def
			/VB \psk@Dosage@VolumeBase\space def
			/VA \psk@Dosage@VolumeAcide\space def
			/KA 10 \psk@Dosage@pKA\space neg exp def
			/pKA \psk@Dosage@pKA\space def
			/pH1 \@nameuse{psk@Dosage@pH1}\space def
			/KE 1e-14 def
			/pHmax KE
			16 CB mul
			CA VA mul
			sub
			16 VA add
			div
			div
			log neg def
			/pHmin
			KA KA mul
			4 CA mul
			KA mul
			add
			sqrt
			KA sub
			2
			div
			log neg def
			/calc {/pH ED
				/H 10 pH neg exp def
				/V CA VA mul
				1 H KA div add
				div
				VA KE H div H sub mul
				add
				CB H add KE H div sub
				div def} def
			/pHE 7 pKA 2 div add
			0.5 CA CB mul CA CB add div log mul add def
			/VE CA VA mul CB div def
			/VE VE 100 mul round 100 div def
			/pHE pHE 100 mul round 100 div def
			\Tangentes
			VE pHE){E}%
			\psclip{\psframe[linestyle=none](\Vbmax,14)}%
			\parametricplot[style=\psk@Dosage@pHstyle]{pHmin}{\myphmax}{%
				t calc
				V t
			}
			\ifPst@dpH
				\parametricplot[style=\psk@Dosage@dpHstyle,plotpoints=400,yunit=\psk@Dosage@dpHunit]{pHmin}{\myphmax}{%
					t calc
					/V1 V def
					%
					t dt add calc
					/V2 V def
					V1 dt V2 V1 sub div
				}
			\fi
			\ifPst@tangentes
				{\psset{style=\psk@Dosage@tangentesstyle}
				\psline(!  v11 ph11)(! v12 ph12)
				\psline(!  v21 ph21)(! v22 ph22)
				\psline(!  vE1 phE1)(! vE2 phE2)
				\psdot(!V1 pH1)
				\psdot(!V2 pH2)}%
			\fi
			\endpsclip
			\ifPst@Equivalence
				\ifPst@values
					\uput[-45](E){\pnode(! \Valeurs
						[/VE= VE /mL /, /pHE= pHE]
						AfficheValeurs 0 0){AA}}
				\fi
				\psdot(E)
				\uput[180](E){E}
			\fi
			%\endpspicture
		}}
%%%%%%%% dosage d'une base faible par un acide fort%%%%%%%%%%%%
\def\dosageBfAF{\@ifnextchar[{\pst@dosageBfAF}{\pst@dosageBfAF[]}}
\def\pst@dosageBfAF[#1]{{%
			\psset{#1}%
			%\pspicture(16,15)
			\graphpaper
			\uput[-90](\Vbmax,0){$\SI[parse-numbers=false]{V_A}{mL}$}
			\ifPst@dpH
				\psline[linecolor=black]{->}(16,0)(16,14)
				\uput[0](\Vbmax,14){$\displaystyle\dv{\pH}{V_A}$}
				% \uput[0](16,14){\Cadre{\sf\white $\displaystyle\mathsf{\frac{dpH}{dV_A}}$}}
			\fi
			\pnode(!%
			/CA \psk@Dosage@ConcentrationAcide\space def
			/CB \psk@Dosage@ConcentrationBase\space def
			/VB \psk@Dosage@VolumeBase\space def
			/KB 10 \psk@Dosage@pKB\space neg exp def
			/pKB \psk@Dosage@pKB\space def
			/pH1 \@nameuse{psk@Dosage@pH1}\space def
			/KE 1e-14 def
			/pHmax 14
			pKB CB log  sub 2 div sub def
			/pHmin
			CA 16 mul
			CB VB mul
			sub
			16 VB add
			div
			log neg def
			/pHE 7 0.5 pKB mul sub 0.5 CA CB mul CA CB add div log mul sub def
			/VE CB VB mul CA div def
			/VE VE 100 mul round 100 div def
			/pHE pHE 100 mul round 100 div def
			/calc {/pH ED
				/H 10 pH neg exp def
				/O KE H div def
				/V O VB mul CB VB mul KB mul O KB add div sub KE VB mul O div sub
				KE O div O sub CA sub
				div def
			} def
			\Tangentes
			VE pHE){E}%
			\psclip{\psframe[linestyle=none](16,14)}%
			\parametricplot[style=\psk@Dosage@pHstyle]{pHmin}{pHmax}{%
				t calc
				V t
			}
			\ifPst@dpH
				\parametricplot[style=\psk@Dosage@dpHstyle,plotpoints=400,yunit=\psk@Dosage@dpHunit]{pHmin}{pHmax}{%
					t calc
					/V1 V def
					%
					t dt add calc
					/V2 V def
					V1 dt V2 V1 sub div 14 \psk@Dosage@dpHunit\space div add
				}
			\fi
			\ifPst@tangentes
				{\psset{style=\psk@Dosage@tangentesstyle}
				\psline(!  v11 ph11)(! v12 ph12)
				\psline(!  v21 ph21)(! v22 ph22)
				\psline(!  vE1 phE1)(! vE2 phE2)
				\psdot(!V1 pH1)
				\psdot(!V2 pH2)}%
			\fi
			\ifPst@Equivalence
				\ifPst@values
					\uput[-45](E){\pnode(! \Valeurs
						[/VE= VE /mL /, /pHE= pHE]
						AfficheValeurs 0 0){AA}}
				\fi
				\psdot(E)
				\uput[180](E){E}
			\fi
			\endpsclip
			%\endpspicture%
		}}
%%%%%%%% dosage d'un polyacide %%%%%%%%%
\def\dosagetriacide{\@ifnextchar[{\pst@dosagetriacide}{\pst@dosagetriacide[]}}
\def\pst@dosagetriacide[#1]{{%
			\psset{#1}%
			\graphpaper
			\uput[-90](\Vbmax,0){$\SI[parse-numbers=false]{V_B}{mL}$}
			\ifPst@dpH
				\psline[linecolor=black]{->}(16,0)(16,14)
				\uput[0](\Vbmax,14){$\displaystyle\dv{\pH}{V_B}$}
				% \uput[0](16,14){\Cadre{\sf\white $\displaystyle\mathsf{\frac{dpH}{dV_B}}$}}
			\fi
			\pnode(!%
			/CA \psk@Dosage@ConcentrationAcide\space def
			/CB \psk@Dosage@ConcentrationBase\space def
			/VB \psk@Dosage@VolumeBase\space def
			/VA \psk@Dosage@VolumeAcide\space def
			/KA 10 \psk@Dosage@pKA\space neg exp def
			/KB 10 \psk@Dosage@pKB\space neg exp def
			/pKA \psk@Dosage@pKA\space def
			/pKB \psk@Dosage@pKB\space def
			/pKA1 \@nameuse{psk@Dosage@pKA1}\space def
			/pKA2 \@nameuse{psk@Dosage@pKA2}\space def
			/pKA3 \@nameuse{psk@Dosage@pKA3}\space def
			/KA1 10 pKA1 neg exp def
			/KA2 10 pKA2 neg exp def
			/KA3 10 pKA3 neg exp def
			/KE 1e-14 def
			/pHmin
			KA1 KA1 mul
			4 CA mul
			KA1 mul
			add
			sqrt
			KA1 sub
			2
			div log neg def
			/pHmax
			KE
			CB 20 mul
			CA 3 VA mul mul
			sub
			20 VA add
			div
			div
			log neg def
			/calc {/pH ED
				/H 10 pH neg exp def
				/vB
				1
				KA2 2 mul H div add
				KA2 KA3 mul 3 mul H H mul div
				add
				CA mul
				H KA1 div 1 add
				KA2 H div add
				KA2 KA3 mul H H mul div add
				div
				KE H div H sub add
				H KE H div sub CB add div
				VA mul def} def
			/pHE1 0.5 pKA1 pKA2 add
			1 KA1 CA CB add CA CB mul div mul add log add mul
			100 mul round 100 div def
			/VE1 CA VA mul CB div
			100 mul round 100 div def VE1 pHE1){E1}%
			\pnode(! /pHE2 0.5 pKA2 pKA3 add
			1 KE KA3 div 2 CA mul CB add CA CB mul div mul add log sub mul
			100 mul round 100 div def
			/VE2 2 CA mul VA mul CB div 100 mul round 100 div def VE2 pHE2){E2}%
			\psclip{\psframe[linestyle=none](16,14)}%
			\parametricplot[style=\psk@Dosage@pHstyle]{pHmin}{pHmax}{%
				t calc
				vB t
			}
			\ifPst@dpH
				\parametricplot[style=\psk@Dosage@dpHstyle,plotpoints=400,yunit=\psk@Dosage@dpHunit]{pHmin}{pHmax}{%
					t calc
					/V1 vB def
					%
					t dt add calc
					/V2 vB def
					V1 dt V2 V1 sub div
				}
			\fi
			\ifPst@Equivalence
				\ifPst@values
					\uput[-45](E1){\pnode(! \Valeurs
						[/VE1= VE1 /mL /, /pHE1= pHE1]
						AfficheValeurs 0 0){AA}}
					\uput[-45](E2){\pnode(! \Valeurs
						[/VE2= VE2 /mL /, /pHE2= pHE2]
						AfficheValeurs 0 0){AA}}
				\fi
				\psdot(E1)
				\uput[180](E1){E1}
				\psdot(E2)
				\uput[180](E2){E2}
			\fi
			\endpsclip
			%\endpspicture
		}}
\catcode`\@=\PstAtCode\relax
\endinput
